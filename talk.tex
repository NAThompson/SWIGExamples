\documentclass{beamer}
\usepackage{listings}
\usepackage{hyperref}
\hypersetup{
colorlinks=true,
urlcolor=blue
}

\lstset{basicstyle=\ttfamily\footnotesize,breaklines=true}

\begin{document}
\title{SWIG}
\author{Nick Thompson} 
\date{\today} 

\frame{\titlepage} 


\begin{frame}[fragile]
\frametitle{Installation}

\begin{lstlisting}[language=bash]
sudo apt-get -y install libpcre3-dev byacc yodl
git clone --depth 1 https://github.com/swig/swig.git
cd swig
./autogen.sh
./configure --prefix=/place/for/swig # or default: /usr/local/share
make -j8 && make install
\end{lstlisting}
\end{frame}

\begin{frame}[fragile]
  \frametitle{Clone the examples}
  \begin{lstlisting}[language=bash]
    git clone https://github.com/NAThompson/SWIGExamples.git
  \end{lstlisting}
\end{frame}

\begin{frame}[fragile]
\frametitle{Wrapping C is Easy:}
\begin{lstlisting}[language=bash]
  cd wrap_pure_c;
  make
  ./say_hello.py
\end{lstlisting}
\end{frame}

\begin{frame}[fragile]
  \frametitle{Steps to Wrap C code}
  \begin{itemize}
  \item Compile the C code into a shared object.
  \item Use ctypes to load the dll (.so)
  \item Tell the Python interpreter how to interpret the output
  \end{itemize}
\end{frame}


\begin{frame}[fragile]
  \frametitle{Let's try to wrap some C++ without SWIG}
  \begin{lstlisting}[language=bash]
    cd wrap_cpp_wo_swig;
    make
    ./say_hello.py
  \end{lstlisting}
  %Name mangling is a pain!
  
  %Also, note how much the Python interpreter cares about your private methods!
\end{frame}

\begin{frame}[fragile]
  \frametitle{Steps:}
  \begin{itemize}
  \item Compile the C++ into a shared object
  \item Determine the mangled symbol name using nm (which is not guaranteed to be consistent across compilers!)
  \item Use ctypes to interpret the return type (which might not be a primitive C type!)
  \item Call the function (which might segfault as C++ does not have a well-defined application binary interface)
  \end{itemize}

  But you can call private methods using this technique . . .
\end{frame}


\begin{frame}
  \frametitle{Using Mangled Symbols in ctypes is Untenable!}
  But note the \texttt{extern ``C''} trick which puts demangled symbols in the symbol table of the shared object.

  Use this method if you can, but note the extern ``C'' rules (can't be in a namespace. . . infinitude of others that will cause cryptic linker errors.)
\end{frame}

\begin{frame}
  \frametitle{How would \emph{you} design a wrapper script for interfacing Python and C++?}
  \begin{itemize}
    \item C variables are passed by pointer or value.
    \item C++ variables can be pass by pointer, value, reference, and rvalue reference, and const qualified to your heart's delight.
    \item Python variables are passed by assignment (creation of references)
  \end{itemize}

\end{frame}

\begin{frame}[fragile]
  \frametitle{Design of SWIG: Hammer the C++ into C code the Python interpreter can use}
  \begin{itemize}
  \item Everything in Python is treated by the Python interpreter as a PyObject.
  \item A PyObject ``is a type which contains the information Python needs to treat a pointer to an object as an object''.
  \item So if we want to wrap a $\texttt{bool foo()}$, we create a new function returning $\texttt{PyObject}$ of type $\texttt{Py\_True}$ or $\texttt{Py\_False}$ . . . which are themselves PyObject*.
  \item Generate code that we can use $\texttt{extern ``C''}$ on; build a Python extension module.
  \item Will it work with Jython? (methinks no . . .)
  \end{itemize}
\end{frame}

\begin{frame}[fragile]
  \frametitle{An example \texttt{PyObject}: \texttt{Py\_True}}
  \begin{itemize}
  \item On my machine, found definition found in: /usr/include/python3.4/boolobject.h
  \item Demonstrates use of garbage collection by reference counting: \texttt{Py\_INCREF(Py\_True)}
  \item Following through to the \texttt{PyObject}, we see that it is a struct which acts essentially as an abstract base class for all Python types (see object.h)
  \end{itemize}
    
\end{frame}

\begin{frame}[fragile]
  \frametitle{SWIG Minimal Working Example}
  \begin{lstlisting}[language=bash]
    cd swig_mwe;
    swig_mwe$ make
    swig_mwe$ python3
    >>> import is_even
    >>> dir(is_even)
    ['__builtins__', '__cached__', '__doc__', '__file__', '__loader__', '__name__', '__package__', '__spec__', '_is_even', '_newclass', '_object', '_swig_getattr', '_swig_getattr_nondynamic', '_swig_property', '_swig_repr', '_swig_setattr', '_swig_setattr_nondynamic', 'is_even']
    >>> is_even.is_even(5)
    False
  \end{lstlisting}
\end{frame}

\begin{frame}[fragile]
  \frametitle{How does it work?}
  \emph{Please remain calm.}
  \begin{lstlisting}[language=c++]
    #ifdef __cplusplus
    extern ``C'' {
      #endif
      SWIGINTERN PyObject *_wrap_is_even(PyObject *SWIGUNUSEDPARM(self), PyObject *args) {
        PyObject *resultobj = 0;
        int arg1 ;
        int val1 ;
        int ecode1 = 0 ;
        PyObject * obj0 = 0 ;
        bool result;

        if (!PyArg_ParseTuple(args,(char *)''O:is_even'',&obj0)) SWIG_fail;
        ecode1 = SWIG_AsVal_int(obj0, &val1);
        if (!SWIG_IsOK(ecode1)) {
          SWIG_exception_fail(SWIG_ArgError(ecode1), ``in method ''' ``is_even'' ``', argument `` ``1" of type ''' ``int"''');
        }
        arg1 = static_cast< int >(val1);
        result = (bool)is_even(arg1);
        resultobj = SWIG_From_bool(static_cast< bool >(result));
        return resultobj;
        fail:
        return NULL;
      }
    \end{lstlisting}  
\end{frame}

\begin{frame}[fragile]
  \frametitle{Steps:}
  \begin{itemize}
  \item Compile the source into a shared object.
  \item Write a \texttt{foo.i} file, which tells swig what to wrap.
  \item Use swig to generate a \emph{4000 line C++/C wrapper file} which demangles all symbols and casts everything into the appropriate inheritor of \texttt{PyObject}, as well as a Python script to load the module
  \item Compile the C++/C wrapper into a shared object (same name as the source .so, with leading underscore)
  \end{itemize}
\end{frame}

\begin{frame}[fragile]
  \frametitle{Obvious in retrospect . . .}
  You cannot compile your extension module linking against PythonX.Y and run it in PythonU.V! You code will work with one and only one Python interpreter, for each build.
\end{frame}



\begin{frame}[fragile]
  \frametitle{SWIG And C++ namespaces}
  \begin{lstlisting}[language=bash]
    cd swig_namespace;
    swig_namespace$ make
    swig_namespace$ python3
    >>> import is_even
    >>> dir(is_even)
    ['__builtins__', '__cached__', '__doc__', '__file__', '__loader__', '__name__', '__package__', '__spec__', '_is_even', '_newclass', '_object', '_swig      _getattr', '_swig_getattr_nondynamic', '_swig_property', '_swig_repr', '_swig_setattr', '_swig_setattr_nondynamic', 'is_even']
    >>> is_even.is_even(5)
    False
  \end{lstlisting}
\end{frame}

\begin{frame}
  \frametitle{Extern C and namespaces}
  \begin{itemize}
    \item Since C doesn't have namespaces, we must tell swig how to call the C++ code, within a namespace, by appropriate editing of foo.i.

    \item Extern C can cause namespace clashes! (There was a reason for introducing this annoyance . . .)
    \item Namespace clashes caused by swig's use of extern C must be resolved by the programmer.
  \end{itemize}
\end{frame}

\begin{frame}[fragile]
  \frametitle{SWIG and a C++ namespace clash}
  ``If your program utilizes thousands of small deeply nested namespaces each with identical symbol names, well, then
  you get what you deserve.`` --Swig documentation
  \begin{lstlisting}[language=bash]
    cd swig_namespace_clash
    swig_namespace_clash$ make
    swig_namespace_clash$ python3 -q
    >>> import is_even
    >>> dir(is_even)
    ['__builtins__', '__cached__', '__doc__', '__file__', '__loader__', '__name__', '__package__', '__spec__', '_is_even', '_newclass', '_object', '_swig_getattr', '_swig_getattr_nondynamic', '_swig_property', '_swig_repr', '_swig_setattr', '_swig_setattr_nondynamic', 'bar_is_even', 'foo_is_even']
    >>> is_even.bar_is_even(4)
    True
    >>> is_even.foo_is_even(4)
   \end{lstlisting}
\end{frame}

\begin{frame}[fragile]
  \frametitle{SWIG and C++ namespace clashes}
  \begin{itemize}
    \item Resolve namespace clashes in your shared object using the ``rename'' functionality in SWIG (see swig\_namespace\_clash/is\_even.i).
  \end{itemize}
\end{frame}

\begin{frame}[fragile]
  \frametitle{How to wrap C++ templates}

  \begin{itemize}
    \item Python doesn't use templates; nor does C. C++ templates are turned into assembly code at compile time, iff there exists a template instantiation.
    \item So you must instantiate the templates to interface C++ and Python.
      \item Note the introduction of \texttt{\%include ``std\_vector.i''} in \texttt{example.i}. This is because it's too complicated for an autogenerated wrapper.
        \end{itemize}
  \begin{lstlisting}[language=bash]
    cd swig_vector;
    make
  \end{lstlisting}

\end{frame}

\begin{frame}[fragile]
  \frametitle{How to wrap a C++11 move constructor:}
  \begin{lstlisting}[language=bash]
    cd wrap_rvalue;
    
  \end{lstlisting}

  You can't! You need to ignore it.
\end{frame}

\begin{frame}[fragile]
  \frametitle{How to get your numpy vectors down to double pointers}
  \begin{lstlisting}
    cd swig_numpy;
    make
    ./example_use.py
  \end{lstlisting}
  See \href{http://wiki.scipy.org/Cookbook/SWIG_NumPy_examples}{SciPy Cookbook}
\end{frame}


\end{document}
